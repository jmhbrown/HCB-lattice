\documentclass[a4paper,10pt]{article}
\usepackage[utf8x]{inputenc}

%opening
\title{Non-Interacting Bosons on an Optical Lattice Subject to a  Time-Dependent Potential}
\author{Jennifer Brown\\
\texttt{jmb347@georgetown.edu}}
\date{\today}

\begin{document}

\maketitle

\begin{abstract}
We extend techniques used to measure observables on a system of non-interacted bosons under a time-independednt trapping potential to the case of a time-dependent potential. 
This is accomplished by approximating the system's hamiltonian as a combination of its time-independent non-diagonal kinetic and time-dependent diagonal parts using a Trotter approximation.
\end{abstract}

\section{Introduction}
Ultracold gases provide window into the paradoxical and often opaque world of quantum mechanics. At unimaginably low temperatures, some bosonic gases form what are call Bose-Einstein condensates. A fine measure of control is brought  over these tamed many-body quantum systems by their confinement in a lattice under a trapping potential. These lattice structures, constructed optically, allow experimentalists to precisely control the hamiltonian of a system. In this paper, we address the difficulties of measuring the time-evolving state of a time-depenedent hamiltonian, when the relationship $|\Phi(t)\>= H|\Phi_i>$ breaks down.

\section{Background}
\subsection{Bose-Einstein Condensates}
Bose-Einstein condensates are an exotic state of matter achieved by some bosonic gases at temperatures approaching absolute zero. Confining these condensates to a lattice and subjecting them to a trapping potential is a powerful method for controlling their behavior on a fine-grain level. These optical lattices provide a tool for exploring fundamental issues in condensed matter physics and allow physicsts to simulate more complicated systems.
\subsection{Optical Lattices}
\subsection{The Trotter 
\section{Implimentation}

\section{Results}

\end{document}
